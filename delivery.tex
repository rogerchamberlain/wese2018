\section{Instructional Delivery}
\label{sec:delivery}

\subsection{Active Learning}

For the last decade, the department has had a stong commitment to
active learning~\cite{scbggg10,sgcggt10}.
The evidence for the benefits of active learning is
extensive~\cite{jjs98,lst99,Prince04,rss97},
and while actively learning is applied broadly, it is particularly
compelling in science, math, and engineering
design~\cite{Freeman14,lst99,Hake98,Byerley01,kb06}, including computer
science~\cite{McConnell96,tlb01,skltc10,ag13}
and cyber-physical systems~\cite{me14,mmy16}.

Active learning is often associated with a \emph{flipped} classroom, in
which students work on problems during the time that would otherwise be
relegated to lecture.  A significant problem with the flipped classroom is
that students make progress on a posed problems at different
rates.
The active learning approach we use at Washington University across the entire
first year is \emph{studio}-based~\cite{hnc08}.
Our studio sessions evolved from observation of our colleagues in
Art and Architecture as they work with their students.  Elements present in
these sessions include collaboration, continual feedback about product
and process, a group of 2-4 students that work at a similar rate, and
a problem to be solved by that group that lacks a unique
answer or approach.  The sessions are guided but unscripted, with the
students in a given group progressing at their own pace.

Collaboration is important to our students' studies because our discipline
is practiced collaboratively.  

\FIXME{RKC left off here}

\FIXME{Describe studio-based instruction as practiced at WU.}

\subsection{Student Assessment}

The assessment of students takes a number of forms, including studio
participation, on-line quizzes, assignments, and exams.
For an individual module on a typical week,
students are expected to watch the videos
and do any required reading prior to the studio session for that module.
A pre-studio quiz (completed on-line) is used to incentivize this
requirement.  For a Monday studio, the first quiz is due Sunday night.
These quiz questions are designed to be straightforward to answer if
they have done what was asked of them (e.g., repetition of facts,
very simple analysis).

Studio itself is not graded based on correct answers to any questions, but
rather participation. The assignment for that module is started on the
following laboratory session (on Wednesday following a Monday studio)
and is due during the laborabory session one week later.
\FIXME{Say something about checkout and rubrics.}

The second quiz is due Wednesday evening, the same day that the assignment
is due. It is also on-line, and for this quiz the questions are designed
to be similar in content, scope, and style as those that they are likely to
experience on the exams. 

At three points during the semester (approximately evenly spaced) there
is a written exam.  During each of those weeks, the
studio session is tailored to have review content, the second laboratory
meeting during the week does not have an assignment due, and the
exam is scheduled in the evening (so that all students are taking the
exam simultaneously).
In some semesters, the last exam has been comprehensive.  For the most
recent semesters, it has been restricted to the last third of the material.
