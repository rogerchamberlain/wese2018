\section{Instructional Delivery}
\label{sec:delivery}

\subsection{Active Learning}

For the last decade, the department has had a strong commitment to
active learning~\cite{scbggg10,sgcggt10}.
The evidence for the benefits of active learning is
extensive~\cite{jjs98,lst99,Prince04,rss97},
and while actively learning is applied broadly, it is particularly
compelling in science, math, and engineering
design~\cite{Freeman14,lst99,Hake98,Byerley01,kb06}, including computer
science~\cite{McConnell96,tlb01,skltc10,ag13}
and cyber-physical systems~\cite{me14,mmy16}.

Active learning is often associated with a \emph{flipped} classroom, in
which students work on problems during the time that would otherwise be
relegated to lecture.  A significant problem with the flipped classroom is
that students make progress on a posed problems at different
rates.
The active learning approach we use at Washington University across the entire
first year is \emph{studio}-based~\cite{hnc08}.
Our studio sessions evolved from observation of our colleagues in
Art and Architecture as they work with their students.  As elaborated below,
elements present in
these sessions include collaboration, continual feedback about product
and process, groups comprised of of 2-4 students who work at a similar rate, 
and a problem to be solved by that group that lacks a unique
answer or approach.  The sessions are guided but unscripted, with the
students in a given group progressing at their own pace.

Collaboration is important to our students' studies for the following reasons.
Studies~\cite{Krause:2012:EFL:2157136.2157192} have shown that women are
attracted in greater numbers to the study of computer science when they
appreciate the extent to which it is practiced collaboratively.  Because
the practice of computer science is largely collaborative, featuring collaboration
in our early courses provides a more accurate impression of computer science
practice to all of our students.  We therefore have intentionally collaborative
sessions that allow our students to enjoy learning by interacting with others
and to develop the skills that are useful for working in groups.

Continual feedback during studio sessions is important to keep groups on
track and to reinforce that the material they are learning is important to
their studies.  For our courses, the feedback and guidance is provided by
a teaching assistant (TA) assigned to each group and by faculty who 
roam between the groups.  The TA is close in age to the students in the
course, typically one or two years further in computer science studies.
Studies have shown that novices are more likely to predict the difficulty
of a task for other novices than are experts~\cite{Hinds:1999}.  Moreover,
the TA is more likely to empathize with the difficulty of a task than is
an expert, and students appreciate the guidance of our TAs as well as their
empathy with the difficulty of studies in computer science.

The studio groups are arranged by the students themselves, often determined
by the way students seat themselves in the room. In our experience, a group
functions best when its members are closest in ability, knowledge, and
experience.  An opposite approach would form a team with some students who
are capable enough to help the other students in the team.  We have found
that resentment develops by all members in such situations, so we hope for
teams with students at the same level.   We do not take steps directly to
form such teams.  Instead, we encourage students to move between teams in
the first few sessions, and they usually settle in a team that is comfortable
for them.  Another important aspect of team work in studio is that we neither
insist upon nor do we expect that teams should reach the end of all the work
we assign in studio.  The idea is to plow deep, not far, and we grade
students' studio sessions on participation, 
not on reaching any particular point in the
studio work.  We encourage students to complete the studio work they did not
finish in the 90-minute studio session outside our watch, 
on their own or with team members,
as they prefer.

Finally, the types of problems we pose in studio session are purposefully
amenable to multiple solutions and approaches.  Examples include designing
a flag and anthem for a fictitious country, determining how to draw a Sierpinski
triangle, and throwing darts randomly at a dartboard to compute an approximation
of $\pi$. Students know there is more than one way to solve any of these
problems, which liberates them from finding the \emph{right} way and
empowers them to find their \emph{own} way.  This lack of structure can make
the sessions challenging for our TAs, but they enjoy watching students work
together to create solutions in studio.


\subsection{Student Assessment}

The assessment of students takes a number of forms, including studio
participation, on-line quizzes, assignments, and exams.
For an individual module on a typical week,
students are expected to watch the videos
and do any required reading prior to the studio session for that module.
A pre-studio quiz (completed on-line) is used to incentivize this
requirement.  For a Monday studio, the first quiz is due Sunday night.
These quiz questions are designed to be straightforward to answer if
they have done what was asked of them (e.g., repetition of facts,
very simple analysis).

Studio itself is not graded based on correct answers to any questions, but
rather participation. The assignment for that module is started on the
following laboratory session (on Wednesday following a Monday studio)
and is due during the laboratory session one week later.
\FIXME{Say something about checkout and rubrics.}

The second quiz is due Wednesday evening, the same day that the assignment
is due. It is also on-line, and for this quiz the questions are designed
to be similar in content, scope, and style as those that they are likely to
experience on the exams. 

At three points during the semester (approximately evenly spaced) there
is a written exam.  During each of those weeks, the
studio session is tailored to have review content, the second laboratory
meeting during the week does not have an assignment due, and the
exam is scheduled in the evening (so that all students are taking the
exam simultaneously).
In some semesters, the last exam has been comprehensive.  For the most
recent semesters, it has been restricted to the last third of the material.
