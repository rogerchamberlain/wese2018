\section{Instructional Delivery and Materials}
\label{sec:delivery}

\subsection{Contact Time}

The class follows the \emph{flipped classroom} paradigm.  The course is lecture-free, with students meeting twice a week in the instructional laboratories.   One of the weekly meetings is an active learning \emph{studio session}.  The other meeting serves multiple needs: it's used for live-grading and in-person feedback on assignments, it's used as a convenient meeting time for group work, and it's also a time when all students should be able to meet with instructional staff for additional assistance with concepts or assignments.

For the last decade, the department has had a strong commitment to active learning~\cite{scbggg10,sgcggt10}. The evidence for the benefits of active learning is extensive~\cite{jjs98,lst99,Prince04,rss97}, and while actively learning is applied broadly, it is particularly compelling in science, math, and engineering design~\cite{Freeman14,lst99,Hake98,Byerley01,kb06}, including computer science~\cite{McConnell96,tlb01,skltc10,ag13}, computer engineering~\cite{hmdpa04,sr02}, and cyber-physical systems~\cite{me14,mmy16}.

Active learning is often associated with a flipped classroom.  In a flipped classroom contact time is spent \emph{actively} engaging with the course content, typically via applied problem solving or structured discussions. Whereas a significant amount of non-contact time is devoted to the content typically delivered via lecture, i.e., introducing conceptual material through either required readings or pre-recorded lecture videos.

A significant problem with the flipped classroom is that students make progress on posed problems at different rates. The approach we use across the entire first year is \emph{studio}-based~\cite{hnc08}. Our studio sessions evolved from observations by our colleagues in Art and Architecture who have a well established studio-based approach.  As elaborated below, elements present in these sessions include collaboration of 2-4 students with comparable work rates, continual feedback about product and process, and a problem to be solved by that group that lacks a unique answer or approach.  The sessions are guided but unscripted, with the students in a given group progressing at their own pace.

Collaboration like that used in the studio sessions may be particlarly well suited to attracting and retaining members of underrepresented groups. Studies~\cite{Krause:2012:EFL:2157136.2157192} have shown that women are attracted in greater numbers to computing when they appreciate the extent to which it is practiced collaboratively.  Moreover, because computing is largely collaborative, featuring collaboration early in the curriculum provides a more accurate representation to all students.  Consequently, studio sessions are intentionally collaborative and allow our students to enjoy learning by interacting with others and to develop group work skills.

Continual feedback during studio sessions is important to keep groups on task and to reinforce the importance of the material.  Feedback is given by the instructional staff, which include instructors as well as undergraduate Teaching Assistants (TAs). Courses are large (4 sections with 100+ students per section), but student-to-staff ratio is kept under 10 to 1 via a large number of TAs.  The bulk of TA work is logged during scheduled class sessions, where a TA is typically assigned to specific groups of students while instructors roam between the groups.  The use of primarily \emph{undergraduate} TAs may have some advantages.  Since many of the TAs have recently completed the same course, they have both recent exposure to the material and empathy for the challenges faced by their peers.  In addition, studies have shown that novices are more likely to predict the difficulty of a task for other novices than are experts~\cite{Hinds:1999}.

The studio groups are arranged by the students themselves, often determined by the way students seat themselves in the room. In our experience, a group functions best when its members are closest in ability, knowledge, and experience.  An opposite approach would form a team with some students who are capable enough to help the other students in the team.  We have found that resentment develops by all members in such situations, so we hope for teams with students at the same level.   We do not take steps directly to form such teams.  Instead, we encourage students to move between teams in the first few sessions, and they usually settle in a team that is comfortable for them.  Another important aspect of team work in studio is that we neither insist upon nor do we expect that teams should reach the end of all the work we assign in studio.  The idea is to plow deep, not far, and we grade students' studio sessions on participation, not on reaching any particular point in the studio work.  We encourage students to complete the studio work they did not finish in the 90-minute studio session outside our watch, on their own or with team members, as they prefer.

% TODO: Two thoughts about the above:  1) I'm not sure that we consistently encourage trying different groups (this should probably be stricter part of the instruction) and 2) Part of the above seems anicdotal --- can we provide supporting evidence?

Finally, the types of problems we pose in studio session are purposefully amenable to multiple solutions and approaches.  Examples from the first semester course include designing a flag and anthem for a fictitious country, determining how to draw a Sierpinski triangle, and throwing darts randomly at a dartboard to compute an approximation of $\pi$. Examples from this course include developing an algorothm to when a person has taken a step from accelerometer data and developing and implementing protocols to exchaning multiple types of messages between computers. Students know there is more than one way to solve any of these problems, which liberates them from finding the \emph{right} way and empowers them to find their \emph{own} way.  There are some potential downsides to this open ended structure.  First, in order to be able to provide assistance, instructional staff need to understand the unique approach being taken.  This, however, provides an opportunity for students in the course to practice communicating course content.   Second, some students are initially uneasy with the open-ended nature of the problems and are concerned about completing the task and finding the \emph{right} solution.  Providing cues that students should focus on process and understanding content rather than end product often helps alleviate these concerns.


\subsection{Tools and Platforms}

The execution platforms used are built around the Arduino Uno~\cite{arduino}, which has an 8-bit AVR microcontroller as the processor, and a traditional desktop (or laptop) machine programmed using Java (with the Eclipse IDE~\cite{eclipse}).

There are several advantages to using the Arduino platform.  First, it has a huge following in the maker/hobbyist community, which provides ample alternative sources of material and can motivate students to pursue extra-curricular projects.  Second, the simple 8-bit microcontroller at its core is conducive to learning basic computer architecture and machine/assembly language, Third, the Arduino IDE is crossplatform, so students can pursue work on their personal computers. Finally, although the platform is based on C++, a language unfamaliar to most students, the C++ conventions used by the platform make it easily accessible to students with Java experience, which was used in the prior course.

Java is also advantageous.  First, students in the course already have significant Java experience based on the prerequsite course.  Second, like Java is cross platform, again, students can work on their computers.  Finally, Java has a large set of libraries that leveraged as well as provide practice working with existing APIs.

While there are a plethora of texts available within the Arduino ecosystem, they  primarily target (and therefore better suited for) the hobbyist community.  Consequently, we have authored a text~\cite{cc17} for use with the course (which has the obvious advantage of being well matched to the course goals).

\subsection{Weekly Activities}

During a typical weekly module, students:
\begin{enumerate}
  \item prepare for studio sessions, by watching videos and/or completing required readings (prior to Monday),
  \item complete an on-line pre-studio quiz, which is intended to incentivize studio preparation but which only tests shallow knowledge from the module (prior to midnight Sunday),
  \item participate in the studio session (class Monday),
  \item complete and do a live demo of an assignment from the previous module (class Wednesday),  and
  \item complete an on-line summary quiz of the previous module (by midnight Wednesday). Summary quiz questions are more subtantial than pre-studio quizzes and are often comparable to exam questions.
\end{enumerate}


\subsection{Student Assessment}

The assessment of students takes a number of forms, including studio participation (15\%), on-line quizzes (15\%), assignments (35\%), and three exams (10\% each).
% TODO: BSIEVER: Check the above numbers

Due to the open-ended nature of studio sessions and the desire to emphasize process and understanding, studio sessions are graded based on  participation.

Assignments are graded based on a rubric.  Students are provided with a summary of the major points in the rubric, but details such as specific testing procedures are often not provided.  The rubrics often evaluate functionality of a submission (does it work?) as well as design decisions (did they choose an appropriate approach?) and implementation details (coding style, etc.). Our TAs are then provided with a rubric that may include finer granularity (i.e., specific points based on appropriate code indentation) or specific testing procedures. Students must show their work to a TA who will complete the rubric and submit the grade for that assignment.

At three points during the semester (approximately evenly spaced) there is a written exam.  During each of those weeks, the studio session is tailored to have review content, the second laboratory meeting during the week does not have an assignment due, and the
exam is scheduled at a common time for all studnets.  In some semesters, the last exam has been comprehensive, but currently it merely covers the last third of the material.
