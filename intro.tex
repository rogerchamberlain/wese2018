\section{Introduction}
\label{sec:intro}

In the spring of 2015, the Dept.~of Computer Science and Engineering at
Washington University in St.~Louis decided to make a significant change
in its first year course sequence for both computer science and
computer engineering majors.  While the first course follows, reasonably
closely, the CS1 curriculum from the ACM/IEEE~\cite{cs13}, our second semester
freshman course had a focus on thread-based parallelism.  The faculty
had collectively decided that the material in that course should be moved
to a later year (it is now offered as a sophomore-level technical elective),
and that a new course should be developed with a focus on how computers
interact with the physical world.
The new course was offered concurrently with the previous course in the
fall of 2015, and has been offered every semester since.

This paper describes that course: what we teach, how we teach it, and
what we have learned in the process.
\FIXME{Describe the highlights, including the core intellectual material,
the Arduino execution platform (which has a big hobbiest community),
the fact that it is studio-based, and a mention of the text~\cite{cc17},
contrasting with educational material with a hobbiest bent.}

The resulting course is well suited to the (admittedly differing) needs of
both computer science and computer
engineering students (e.g., we believe it is helping students discover
that they want to be computer engineers when they come into the course
not really knowing what computer engineering is).
In addition, we believe it is well positioned as an important course for
any student who wishes to be well educated in cyber-physical systems,
independent of major.
The subject matter is central to the National Academies' report on
cyber-physical systems education~\cite{nasem16}.

Unlike many introductory courses on embedded systems, which are upper-division
courses typically taken during the junior or senior year, this course is
designed to be a freshman-level course, accessible to students in their
first year. With the proliferation of
computing well beyond the desktop and the server room, the sheer numbers
of computers that interact with the physical world are growing and
will continue to grow.
We believe that an early (first-year) introduction to the core concepts
introduced in this course is well deserved.

The organization of the paper is as follows.  Section~\ref{sec:topics}
articulates the intellectial material that we hope to impart on the
students.  
Section~\ref{sec:delivery} describes the educational approach used,
which is a member of the active learning family.
Section~\ref{sec:weeks} provides a brief syllabus, and
Section~\ref{sec:lessons} describes what we have learned as the course
was developed and taught over the course of seven semesters.
Section~\ref{sec:conclude} concludes and describes ideas we intend
to incorporate into future offerints of the course.
