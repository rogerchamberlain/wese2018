\section{Introduction}
\label{sec:intro}

% NOTE: BSIEVER: Adjusted
In the spring of 2015, the Dept.~of Computer Science and Engineering at
Washington University in St.~Louis decided to alter
its first-year course sequence for both computer science and
computer engineering majors.  While the first course follows
the CS1 curriculum from the ACM/IEEE~\cite{cs13}, our second semester
course had a focus on thread-based parallelism.  The faculty
 collectively decided that the material in that course should be moved
to a later year and that a replacement course should focus on
how computers interact with the physical world.
The new course\footnote{The Larsen family's support aided the development of the course.} was offered concurrently with the previous course in the
fall of 2015 and has been offered every semester since.
It is required for all computer science and computer engineering majors,
and is a technical elective to computer science minors and electrical
engineering majors.

This paper describes that course: what we teach, how we teach it, and
what we have learned in the process.  The course content was motivated by the core concepts that are foundational to both computer science and computer engineering while simultaneously being critical to interactions with the physical world.
This includes information representation (e.g., digital inputs
and outputs represented at the bit level, analog input and output values as
non-negative binary integers),
timing (\emph{when} something happens as a functional property),
automata models (finite-state machines), etc.

The course is lecture-free, but meets twice a week in the
instructional laboratories.  One meeting is devoted to studio, where students perform guided exercises through the material we
expect them to master. The other session provides opportunities for assistance on assignments and is used for assessment purposes.

The students work on an Arduino Uno %~\cite{arduino}
,which has an 8-bit microcontroller programmed in C++,
and a traditional computer programmed in Java.
There are two significant advantages to using the Arduino platform.
First, it has a supportive maker/hobbyist community, which can motivate students.
Second, its simplicity is conducive to
learning computer architecture and machine/assembly language.
While there are a plethora of texts available within the Arduino
ecosystem, they primarily target the hobbyist community. Since
our course is intended to focus on core concepts, we have authored
a text~\cite{cc17} for use with the course, which has the obvious
advantage of being well matched to the course goals.

The resulting course is  suited to the common needs of
both computer science and computer engineering students.
In addition, we believe it is well positioned as an important course for
any student who wishes to be well educated in cyber-physical systems,
independent of major.
The subject matter is central to the National Academies' report on
cyber-physical systems (CPS) education~\cite{nasem16}.
Unlike many introductory embedded systems courses, which are upper-division
offerings, this is intentionally
designed to be a first-year course, accessible after only
one semester of introductory computer science.
%The organization of the paper is as follows.  Section~\ref{sec:topics}
%articulates the intellectual material that we hope to impart on the
%students.
%Section~\ref{sec:delivery} describes the educational approach used,
%which is a member of the active learning family.
%Section~\ref{sec:weeks} provides a brief syllabus, and
%Section~\ref{sec:lessons} describes what we have learned as the course
%was developed and taught over the course of seven semesters.
%Section~\ref{sec:conclude} concludes and describes ideas we intend
%to incorporate into future offerings of the course.
