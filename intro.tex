\section{Introduction}
\label{sec:intro}

In the spring of 2015, the Dept.~of Computer Science and Engineering at
Washington University in St.~Louis decided to make a significant change
in its first-year course sequence for both computer science and
computer engineering majors.  While the first course follows, reasonably
closely, the CS1 curriculum from the ACM/IEEE~\cite{cs13}, our second semester
first-year course had a focus on thread-based parallelism.  The faculty
had collectively decided that the material in that course should be moved
to a later year (it is now offered as a sophomore-level technical elective),
and that a new course should be developed with a focus on how computers
interact with the physical world.
The new course was offered concurrently with the previous course in the
fall of 2015, and has been offered every semester since.
It is required for all computer science and computer engineering minors,
and is available as a technical elective to computer science minors (the
department does not offer a computer engineering minor) and electrical
engineering majors.

This paper describes that course: what we teach, how we teach it, and
what we have learned in the process.
The inspiration is that there is significant material that are core
computer science and computer engineering concepts which are well motivated
by the needs of computers as they interact with the physical world.
These include concepts such as information representation (e.g., digital inputs
and outputs represented at the bit level, analog input and output values as
non-negative binary integers with a given range),
timing (when something happens is a functional property of the program),
automata models (finite-state machines), etc.

The course is lecture-free, with students meeting twice a week in the
instructional laboratories.  One meeting per week is studio, in which the
students perform guided exercises aimed at the intellectual material we
expect them to master, and the other meeting is to work on assignments
that will be graded for assessment purposes.

The execution platforms used are built around the Arduino Uno~\cite{arduino},
which has an 8-bit AVR microcontroller as the processor,
and a traditional desktop (or laptop) machine that the
students program using Java (with the Eclipse IDE~\cite{eclipse}).
There are two specific advantages to using the Arduino platform.
First, it has a huge following in the maker/hobbyist community, which
is highly motivational for the students.  We regularly point them to
follow-on opportunities for extra-curricular projects.
Second, the simple 8-bit microcontroller at its core is conducive to
learning basic computer architecture and machine/assembly language
(which we include in the course as an additional core computer
science and engineering concept worthy of exposure in the first year).

There is an advantage to using Java with Eclipse as well. That is the
development environment the students use in the first semester course, so
it is familiar to them already as they enter.  This decreases the chance
of cognitive overload on purely logistical subjects, the Eclipse environment
is familiar, the Arduino development environment is very simple, the
Java language is familiar, and the Arduino C subset is quite close to
Java syntactically.

While there are a plethora of texts available within the Arduino
ecosystem, they are primarily targeting (and therefore better suited for)
the hobbyist community. Since
our course is intended to focus on core concepts, using the
Arduino platform for illustration and experimentation, we have authored
a text~\cite{cc17} for use with the course (which has the obvious
advantage of being well matched to the course goals).

The resulting course is well suited to the (admittedly differing) needs of
both computer science and computer
engineering students (e.g., we believe it is helping students discover
that they want to be computer engineers when they come into the course
not really knowing what computer engineering is).
In addition, we believe it is well positioned as an important course for
any student who wishes to be well educated in cyber-physical systems,
independent of major.
The subject matter is central to the National Academies' report on
cyber-physical systems education~\cite{nasem16}.

Unlike many introductory courses on embedded systems, which are upper-division
courses typically taken during the junior or senior year, this course is
designed to be a first-year course, accessible to students after only
one semester of introductory computer science.  With the proliferation of
computing well beyond the desktop and the server room, the sheer numbers
of computers that interact with the physical world are growing and
will continue to grow.
We believe that an early (first-year) introduction to the core concepts
introduced in this course is well deserved.

The organization of the paper is as follows.  Section~\ref{sec:topics}
articulates the intellectual material that we hope to impart on the
students.  
Section~\ref{sec:delivery} describes the educational approach used,
which is a member of the active learning family.
Section~\ref{sec:weeks} provides a brief syllabus, and
Section~\ref{sec:lessons} describes what we have learned as the course
was developed and taught over the course of seven semesters.
Section~\ref{sec:conclude} concludes and describes ideas we intend
to incorporate into future offerings of the course.
