\section{Introduction}
\label{sec:intro}

In the spring of 2015 the Dept.~of Computer Science and Engineering at Washington University in St.~Louis decided to change the core introductory course sequence, which is required of computer science and computer engineering majors.  Historically the core  consisted of two courses: an introductory programming that corresponds to ACM/IEEE~\cite{cs13} Computer Science curricular guidelines and a subsequent course on concurrency.  The change was motivated by two factors:
\begin{enumerate}
  \item topics in concurrency may be more appropriate for a more mature audience, such as sophomores who may have more programming experience and
  \item the faculty wanted to introduce significant concepts in embedded and cyber-physical systems (CPSearly enough in the curriculum for students to be able to pursue additional depth courses in these areas.
\end{enumerate}

% TODO: BSIEVER: Fact-check;  Is the concurrency course optional for majors/minors?
Consequently, the course in concurrency was changed to an elective course and a new required course was added to the core.  This new course, which can also be taken as a technical elective by computer science minors and electrical engineering majors, was first offered in the fall of 2015.  It fully displaced the prior requirement in the spring of 2016 and has been offered regularly since.  In order to engage this audience, the course uses a flipped classroom, with the majority of contact time spent in studio-based active learning.

Unlike many introductory courses on embedded systems, which are upper-division courses typically taken during the junior or senior year, this course is designed to be a first-year course, accessible to students after only one semester of introductory computer science.  Due to the proliferation of embedded systems, we believe that an early (first-year) introduction to the core concepts in embedded systems is well deserved.

The content of the resulting course is well suited to the differing needs of both computer science and computer engineering students.  We believe it is well positioned as an important course for any students who wish to be well educated in cyber-physical systems, independent of major. Moreover, we hope that it may help students discover that they want to be computer engineers early in the curriculum. Finally, the subject matter is part of the foundational material described in the National Academies' report on cyber-physical systems education~\cite{nasem16}.
% TODO: BSIEVER: Fact-check the "foudnational material" quote above.

This paper describes this course: the content, pedagogical approach, and lessons learned.  The paper is organized as follows: Section~\ref{sec:topics} articulates the intellectual material that we hope to impart on the students. Section~\ref{sec:delivery} describes the environment, audience, educational approach used. Section~\ref{sec:weeks} provides a brief syllabus, and Section~\ref{sec:lessons} describes what we have learned as the course was developed and taught over the course of seven semesters. Section~\ref{sec:conclude} concludes and describes ideas we intend to incorporate into future offerings of the course.







%   Non-intro / introduce
