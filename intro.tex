\section{Introduction}
\label{sec:intro}

In the spring of 2015, the Dept.~of Computer Science and Engineering at
Washington University in St.~Louis decided to alter
its first-year course sequence for both computer science and
computer engineering majors.  While the first course follows
the CS1 curriculum from the ACM/IEEE~\cite{cs13}, our second semester
first-year course had a focus on thread-based parallelism.  The faculty
had collectively decided that the material in that course should be moved
to a later year and that a new course should be developed with a focus on 
how computers interact with the physical world.
The new course was offered concurrently with the previous course in the
fall of 2015, and has been offered every semester since.
It is required for all computer science and computer engineering majors,
and is available as a technical elective to computer science minors and electrical
engineering majors.

This paper describes that course: what we teach, how we teach it, and
what we have learned in the process.
We were inspired by the significant material that are core
computer science and computer engineering concepts which are well motivated
by the needs of computers as they interact with the physical world.
This includes concepts such as information representation (e.g., digital inputs
and outputs represented at the bit level, analog input and output values as
non-negative binary integers),
timing (\emph{when} something happens as a functional property),
automata models (finite-state machines), etc.

The course is lecture-free, meeting twice a week in the
instructional laboratories.  One meeting is studio, in which the
students perform guided exercises aimed at the intellectual material we
expect them to master, and the other meeting is to work on assignments
that will be graded for assessment purposes.

The execution platforms we use are built around the Arduino Uno~\cite{arduino},
which has an 8-bit AVR microcontroller as the processor,
and a traditional computer that the
students program using Java (with the Eclipse IDE~\cite{eclipse}).
There are two specific advantages to using the Arduino platform.
First, it has a huge following in the maker/hobbyist community, which
is highly motivational for the students. 
Second, the simple 8-bit microcontroller at its core is conducive to
learning basic computer architecture and machine/assembly language.
While there are a plethora of texts available within the Arduino
ecosystem, they primarily target the hobbyist community. Since
our course is intended to focus on core concepts, we have authored
a text~\cite{cc17} for use with the course, which has the obvious
advantage of being well matched to the course goals.

The resulting course is well suited to the needs of
both computer science and computer engineering students.
In addition, we believe it is well positioned as an important course for
any student who wishes to be well educated in cyber-physical systems,
independent of major.
The subject matter is central to the National Academies' report on
cyber-physical systems (CPS) education~\cite{nasem16}.
Unlike many introductory embedded systems courses, which are upper-division
offerings, this is intentionally
designed to be a first-year course, accessible to students after only
one semester of introductory computer science.

%The organization of the paper is as follows.  Section~\ref{sec:topics}
%articulates the intellectual material that we hope to impart on the
%students.  
%Section~\ref{sec:delivery} describes the educational approach used,
%which is a member of the active learning family.
%Section~\ref{sec:weeks} provides a brief syllabus, and
%Section~\ref{sec:lessons} describes what we have learned as the course
%was developed and taught over the course of seven semesters.
%Section~\ref{sec:conclude} concludes and describes ideas we intend
%to incorporate into future offerings of the course.
