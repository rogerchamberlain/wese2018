\section{Intellectual Topics}
\label{sec:topics}

The course provides both exposure to and mastery of a variety of concepts
in embedded systems and CPS, which are also significant components
of the ACM/IEEE
Computer Science~\cite{cs13} and Computer Engineering~\cite{ce16} curricula.
Below we list what intellectual topics we cover in the course.
Items marked with an astrisk ($^*$)
indicate topics that students are expected to gain significant mastery
while non-marked items indicate concepts with lesser depth.

\vspace{0.1in}
\noindent
\begin{tabular}{r l}
{\bf 2.1} & {\bf Information Representation} \\
-- & Integer Representations: both unsigned$^*$ and 2's complement$^*$ \\
-- & Hexadecimal and binary representations$^*$ \\
-- & Non-Integer Numbers: Fixed point and IEEE-754 Floating Point \\
-- & Character Representations: ASCII$^*$ and Unicode \\
-- & Character strings, both as objects and as null-terminated C-style strings \\
%  \item Use of fields in binary representations (as in Floating Point)
-- &  Bitwise operations$^*$ \\
%  \item Arrays and memory organization
-- & Representation of real-world values as voltages that require conversion to a\\
\  & more appropriate unit (e.g., voltage to a temperature)$^*$
\end{tabular}

\noindent
\begin{tabular}{r l}
{\bf 2.2} & {\bf Automata}\\
-- & Moore style finite-state machines$^*$
\end{tabular}

\vspace{0.1in}
\noindent
\begin{tabular}{r l}
{\bf 2.3} & {\bf Timing and Events} \\
-- & Using delays in execution (or sleep) to control time-based behavior$^*$ \\
-- & Using delta timing techniques to maintain either a fixed period between \\
\  & actions or fixed rate of actions (assuming a feasible schedule)$^*$ \\
-- & Using time-division multiplexing of a shared signal line \\
-- & The basic structure of an event loop for event-driven programming$^*$
\end{tabular}

\vspace{0.1in}
\noindent
\begin{tabular}{r l}
{\bf 2.4} & {\bf Circuits Principles, Physical I/O, and User I/O} \\
-- & Physical buttons and the concept of ``bounce'' on real digital inputs$^*$ \\
-- & Analog voltages and digitization of voltages (ADC) \\
-- & Pulse Width Modulation \\
-- & Ohm's Law and current constraints/current limiting$^*$ \\
-- & Challenges of noisy data (e.g., step detection in accelerometer data)$^*$ \\
-- & Pixel displays$^*$
\end{tabular}

\vspace{0.1in}
\noindent
\begin{tabular}{r l}
{\bf 2.5} & {\bf Communications} \\
-- & Serial data exchanges$^*$ \\
-- & Non-blocking communication$^*$ \\
-- & Byte-ordering concerns when exchanging multi-byte entities$^*$ \\
-- & Basic protocols and the exchange of tagged records$^*$
\end{tabular}

\vspace{0.1in}
\noindent
\begin{tabular}{r l}
{\bf 2.6} & {\bf Architecture / Assembly Language} \\
-- & Assembly language for integer arithmatic$^*$ \\
-- & Registers and register conventions$^*$ \\
-- & Function call/use protocols$^*$ \\
-- & Stack use conventions$^*$ \\
-- & Memory organization$^*$ \\
-- & C-style memory segmentation (stack, heap, and text segments)
\end{tabular}
