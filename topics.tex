\section{Intellectual Content}
\label{sec:topics}

While the course is aimed at first-year students, it is not intended to be a survey course.  It provides both exposure to and mastery of a variety of concepts in embedded and CPS, which are also significant components of the ACM's Computer Science and Computer Engineering curriculums \cite{cs13, ce16}.  Below we describe what intellectual topics we cover in the course.  Items marked with an astrisk ($^*$)   indicate topics that students are expected to have significant mastery of while non-marked items indicate concepts with lesser depth.

\subsection{Information Representation}
\label{sec:ip}
\begin{itemize}
  \item Integer Representations: Both unsigned$^*$ and Twos compliment$^*$
  \item Hexadecimal and binary representations$^*$
  \item Non-Integer Numbers: Fixed point and IEEE-754 Floating Point
  \item Character Representations: ASCII$^*$ and Unicode
  \item Character strings, both as objects and as null-terminated C-style strings
  \item Use of fields in binary representations (as in Floating Point)
  \item Bitwise operations$^*$
  \item Arrays and memory organization
  \item Representation of real-world values as voltages that require conversion to a more appropriate unit (e.g., voltage to a temperature)$^*$
\end{itemize}

\subsection{Finite-State Machines}
\label{sec:fsm}
\begin{itemize}
\item Moore style finite state machines$^*$ are a fundamental component of multiple assignments.
\end{itemize}

\subsection{Timing and Events}
\label{sec:time}
\begin{itemize}
  \item Using delays in execution (or sleep) to control time-based behavior of devices$^*$
  \item Using delta timing techniques to maintain either a fixed period between actions or fixed rate of actions (assuming a feasible schedule)$^*$
  \item Using time-division multiplexing of a shared signal line
  \item The basic structure of an event loop for event-driven programming$^*$
\end{itemize}

\subsection{Physical I/O and Circuits Principles}
\label{sec:pio}
\begin{itemize}
  \item Physical buttons and the concept of ``bounce'' on real digital inputs$^*$
  \item Analog voltages and digitization of voltages (ADC)
  \item Pulse Width Modulation
  \item Ohm's Law and current constraints/current limiting$^*$
  \item Challenges of noisy data (e.g., step detection in accelerometer data)$^*$
  \item
\end{itemize}

\subsection{Communications}
\label{sec:comm}
\begin{itemize}
  \item Serial data exchanges$^*$
  \item Non-blocking communication$^*$
  \item Byte-ordering concerns when exchanging multi-byte entities$^*$
  \item Basic protocols and the exchange of tagged records$^*$
\end{itemize}

\subsection{Architecture / Assembly Language}
\label{sec:arch}
\begin{itemize}
  \item Assembly language for integer arithmatic$^*$
  \item Registers and register conventions$^*$
  \item Function call/use protocols$^*$
  \item Stack use conventions$^*$
  \item Memory organization$^*$
  \item C-style memory segmentation (Stack, Heap, and Text segments)
\end{itemize}
