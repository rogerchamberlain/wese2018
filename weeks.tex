\section{Modules}
\label{sec:weeks}

The class starts a new module each
non-exam week, completing it the following week. There are minor schedule
adjustments around the exam, but students generally have one calendar week
to complete each module's assignment.
Also, the first laboratory meeting (labeled studio 0) is used to familiarize
the students with various logistics, such as the development environment
(both Eclipse~\cite{eclipse} and Arduino IDE~\cite{arduino}),
the code repository, etc.
The modules are described below.

\emph{Information Representation} (Module A) --
How information is represented in digital systems, including binary and
hexadecimal conversions, integer
number representations (including 2's complement), as well as
ASCII and UTF text.
Introduction to the finite-state machine abstraction,
and how to implement a finite-state machine in software.
%All of the above concepts are reinforced multiple times over the course
%of the semester, mastery is definitely not expected with only one week
%of exposure.

\emph{Microcontroller Platform} (Module B) --
How to physically implement electrical
circuits, a brief introduction to electricity (a physics course is not a
prerequisite) and voltage, digital inputs and outputs, and finite-state
machine design.

\emph{Real-Time Computing} (Module C) --
Techniques for including time as a functional
element in software. Starting with delay-based timing, students progress to delta
timing (checking each loop to determine if the desired time has elapsed).
This is combined with analog inputs (including the conversion of raw A/D
input values into engineering units) and simple filtering (averaging).
%utilizing a temperature probe that provides a 10~mv/$^\circ$C response.

\emph{Communications} (Modules D,E) (2 weeks) --
This pair of modules investigates the issues inherent in using a byte stream
to communicate between two dissimilar computer systems (the Arduino using C
and the desktop PC using Java).  Students explore how
to serialize multi-byte data types, design protocols that enable synchronization
in the face of dropped bytes,
and design finite-state machine recognizers to parse incoming messages.

\emph{Multiplexing} (Module F) --
Introduction to the concept of time-division multiplexing, asking the
students to interface to and author the software to drive a 5 by 7 pixel
LED display. Real-time computing concepts are reinforced, and students
are introduced to the fundamental mechanisms involved in larger, more
sophisticated, pixed-based displays (e.g., font design, time multiplexing).

\emph{Integrative Project} (Module G) --
Design and implemention of an integrative project
that seeks to reinforce concepts that have been introduced earlier in the
semester.  A second analog sensor is made available (an accelerometer)
for the project.
Different projects have been used in different semesters, including
the development of a pedometer and a predator-prey game (using the
accelerometer to sense the tilt of the microcontroller).

\emph{Assembly Language} (Modules H,I,J) (3 weeks) --
This three week set of modules has an introduction to basic computer
architecture
% (what is a fetch-execute engine)
and machine language, and then
focuses on assembly language, using the simple, 8-bit instruction set.
Week~1 covers data representation (especially multi-byte
data) and data manipulation
(e.g., understanding the need for a carry bit).
Week~2 introduces techniques for implementing if-then-else logic
and looping structures.
Week~3 covers pointers and array access.
%At the end of these three modules, students are not expected to be
%sophisticated assembly language programmers by any means.  The motivation
%is primarily to ensure the computer science students have been exposed,
%and the computer engineering students are better prepared for the deeper
%coverage that comes later (e.g., in their computer architecture course).

The relationship between intellectual topics and individual modules is
shown in Table~\ref{tbl:topics}. The columns in the table represent
modules, and the rows represent topics.
There is some overlap between topic names and module labels; however, that
typically implies that the specific topic is central to the module of the
same name, not that it is the only topic present in the module.

\begin{table}[t]
\caption{Relationship between intellectual topics and modules.}
\label{tbl:topics}

\centerline{\mbox{\ }\hspace{2in} Time $\longrightarrow$}
\centering
\begin{tabular}{l | c | c | c | c | c | c | c}
Topic $\Downarrow$ \ \ \ \ \ \ \ \ \ \ \ \ \ \ \ \ \ \ Module $\Rightarrow$ & A & B & C & D,E & F & G & H,I,J \\ \hline
Information Representation (\textsection 2.1) & $\blacksquare$ & $\blacksquare$ & $\blacksquare$ & $\blacksquare$ & & $\blacksquare$ & $\blacksquare$ \\ \hline
Automata (\textsection 2.2) & $\blacksquare$ & $\blacksquare$ & & $\blacksquare$ & & $\blacksquare$ & \\ \hline
Timing (\textsection 2.3) & & $\blacksquare$ & $\blacksquare$ & & $\blacksquare$ & $\blacksquare$ & \\ \hline
Circuits \& I/O (\textsection 2.4) & & $\blacksquare$ & $\blacksquare$ & $\blacksquare$ & $\blacksquare$ & $\blacksquare$ & \\ \hline
Communications (\textsection 2.5) & & & & $\blacksquare$ & & $\blacksquare$ & \\ \hline
Architecture / Assembly Lang. (\textsection 2.6) & & & & & & & $\blacksquare$ \\
\end{tabular}
%\centerline{\mbox{\ }}
%\centerline{Column Legend}
%\centering
%\begin{tabular}{c | l || c | l}
%Module & Name & Module & Name \\ \hline
%A & Information Representation & F & Multiplexing \\
%B & Microcontroller Platform & G & Integrative Project \\
%C & Real-Time Computing & H,I,J & Assembly Language \\
%D,E & Communications \\
%\end{tabular}
\end{table}

In addition to the modules described above, there are a few modules that
have been developed and used in previous semesters, but are not in current use.

\emph{Interrupts} --
Students explore how interrupts literally interrupt currently executing code.  In addition, students see how interrupts improve responsivenes to events, like button presses (which can further excerbate bouncing).

\emph{IP Networking} --
This module introduces students to topics in networking.  IP protocols,
including both TCP/IP and UDP/IP are covered, as well as domain name service
and socket-based communications.
