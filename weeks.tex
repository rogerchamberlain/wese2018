\section{Modules}
\label{sec:weeks}

As described above, each module is comprised of a pre-quiz, studio,
an assignment, and a post-quiz.  The class starts a new module each
non-exam week, completing it the following week. There are minor schedule
adjustments around the exam, but students generally have one calendar week
to complete each module's assignment.  The currently available modules
(both in use currently and developed in previous semesters but not
currently in use) are described below.

\emph{Information Representation} -- Implementing finite-state machines.
Bit-manipulation operations.

\emph{Microcontroller Platform} -- Designing finite-state machines.
Digital input and output.

\emph{Real-Time Computing} -- Analog input. Simple filtering.

\emph{Computer Communications} (2 weeks) -- Protocol design.

\emph{Multiplexing} -- Pixel-based displays.

\emph{Integrative Project} -- Acceleration. Pedometer. Predator-prey game.

\emph{Computer Architecture and Assembly Language} (3 weeks) -- Data layout.
Program control. Arrays in memory.

\FIXME{Available modules not in current use.}

\emph{Interrupts}

\emph{IP Networking}
